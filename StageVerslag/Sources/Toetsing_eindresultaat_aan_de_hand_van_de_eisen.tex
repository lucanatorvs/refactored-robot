\chapter{Toetsing eindresultaat aan de hand van de eisen}
\label{Toetsing_eindresultaat_aan_de_hand_van_de_eisen}
%%%%%%%%%%%%%%%%%%%%%%%%%%%%%%%%%%%%%%%%%%%%%%%%%%%%%%%%%%%%%%%%%%%%%%%%

Door een modem van Zero-ev te gebruiken hoeven we ons geen zorgen te maken over
of het aan de betreffende wet- en regelgeving voldoet, dat moet namelijk
allemaal door Zero-ev zijn geregeld. 

Om de overigen eisen te testen, zijn we met de proefopstelling (zie hoofdstuk
\ref{sec:proefopstelling}) langs verschillende CCS-snelladers gegaan. We hebben
getest of: het laden werkt, er ingesteld kan worden wat de gewenste laadstroom
is en of het laden makkelijk gestart en gestopt kan worden. Ook is de
communicatie tussen het BMS en het modem via de CCS CAN-controller getest.



\begin{figure}[]
    \centering
    \begin{minipage}{0.45\textwidth}
        \centerline{\includegraphics[width=0.9\textwidth]{testen1}}
        \caption{Testen van de CCS proefopstelling bij een Fastned snellader}
        \label{fig:testen1}
    \end{minipage}\hfill
    \begin{minipage}{0.45\textwidth}
        \centerline{\includegraphics[width=0.9\textwidth]{testen2}}
        \caption{Testen van de CCS proefopstelling bij een Allego snellader}
        \label{fig:testen2}
    \end{minipage}
\end{figure}